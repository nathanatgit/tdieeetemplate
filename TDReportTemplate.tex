\documentclass[12pt,journal,onecolumn]{IEEEtran}

%% ** Style Notice **
%% ** Different IEEE style can be chosen by changing \documentclass[12pt,journal,onecolumn]{IEEEtran} to \documentclass[12pt,conference,onecolumn] and etc.
%% ** For detail info please visit IEEE_Original folder and check IEEEtran_HOWTO.pdf
%% ** One Column style with normal font of 12 pt is set, delete onecolumn in doucumentclass if you need double column
%% ** Style is mainly based on IEEEtrans template and if you want to change different line distance or font scale, check IEEEtran_HOWTO.pdf
%% ** Tested in Microsoft Visual Studio Code, with Latex-Workshop extension, texlive2018 environment installed. No guarantee on other platform (linux or mac) or other tex environment.

\usepackage[pdftex]{graphicx}
\DeclareGraphicsExtensions{.pdf,.jpeg,.png,.eps}
\usepackage{svg}
\usepackage{epstopdf}
\usepackage{cite}
\usepackage{amsmath}
\usepackage{algorithmic}
\usepackage{array}
\usepackage{amsfonts,amssymb}
\usepackage{textcomp}
\ifCLASSOPTIONcompsoc
 \usepackage[caption=false,font=normalsize,labelfont=sf,textfont=sf]{subfig}
\else
 \usepackage[caption=false,font=footnotesize]{subfig}
\fi
\usepackage{url}
\usepackage[hidelinks]{hyperref}

\usepackage[UTF8, scheme = plain]{ctex} % Add support for Chinese characters. If you want your \Section title \ Figure Caption\Table Caption \ to be Chinese, use another CN template.

\usepackage[english]{babel} % Incase you have some special french or other characters in your bib and main tex. Comment out if you don't need.

\usepackage{caption} % Reduce the space after caption.
\captionsetup{font=scriptsize}
\setlength\belowcaptionskip{-8pt}

\usepackage[autostyle, english = american]{csquotes} %Resolve quotation mark  direction problems.
\MakeOuterQuote{"}

\usepackage{etoolbox} % Use patchcmd to control alignment of section title.

\hyphenation{semi-conduc-tor} % Avoid unwanted hyphenation.


\begin{document}

% this is the Cover
\begin{center}
  \quad
  
  \vspace{1cm}

  \includegraphics[width=7cm]{./logo/TDFonts.eps}
  \vspace{1cm}


  {
  \fontfamily{phv}
  \fontsize{24}{30}
  \selectfont \textbf{SAMPLE TITLE HERE}
  }

  \vspace{1cm}
  \includegraphics[width=5.5cm]{./logo/TDLogo.eps}
  \vspace{5cm}

  %% ** Warning::
  %% ** The following \songti control sequence only works on Windows, Linux and mac platform please comment it out
  \begin{table}[h!]
    \centering
    \label{tab:authorinfo}
    \fontsize{16}{30}
    \selectfont
    \begin{tabular}{cl}
    {\songti \textbf{学\ \ \ \ 院:}} & {\songti  \textbf{************学院\ \ }} \\ \cline{2-2}
    {\songti \textbf{专\ \ \ \ 业:} }& {\songti \textbf{************}}\\ \cline{2-2}
    {\songti \textbf{姓\ \ \ \ 名:} }& {\songti \textbf{** ****}}\\ \cline{2-2}
    {\songti \textbf{学\ \ \ \ 号:}} & {\songti \textbf{1234567890}}\\ \cline{2-2}
    \end{tabular}
  \end{table}
  {\fontsize{14}{20}
  \selectfont 2019年12月16日}
  
\end{center}
  \thispagestyle{empty}
  \setcounter{page}{0}
  \clearpage

\title{\textbf{SAMPLE TITLE}}

%% This section changes the Footer and Header of page
\markboth{Marine Acoustic Detection}% even 
{Marine Acoustic Detection} % odd
\maketitle

%% Abstract here if you want write abstract, comment it out.
% \begin{abstract}
%   The abstract goes here.
% \end{abstract}

%% Keyword, comment out before use.
% \begin{IEEEkeywords}
%   IEEE, IEEEtran, journal, \LaTeX, paper, template.
% \end{IEEEkeywords}

\patchcmd{\section}{\centering}{\raggedright}{}{} % Change default IEEE template centering section title to align left
\patchcmd{\section}{\normalsize}{\large\bfseries}{}{} % Change Title to large Boldface

\section{Introduction}

"The Islands of Tahiti are a mythical destination and these islands are a universe where dreams meet reality." These attracting descriptions draws the attention into research.\cite{populationref}.

\begin{figure}[!h]
	\centering
	\subfloat[Tahiti's Location on Earth]{
		\includegraphics[height=5cm]{figures/tahitiloc.png}
		\label{tahitiloc}}
	\quad
	\subfloat[Tahiti from space]{
    \includegraphics[height=5cm]{figures/tahitinear.png}
		\label{tahitinear}}
	\quad
	\caption{Tahiti's Location}
	\label{tahitilocation}
\end{figure}

\subsection{Geography}

Subsection goes here. Sample Text.

The island is 45 km (28 mi) across at its widest point and covers an area of 1,045 km2 (403 sq mi). The highest peak is Mont Orohena (2,241 m (7,352 ft)). Mount Roonui, or Mount Ronui (Mou'a Rōnui), in the southeast rises to 1,332 m (4,370 ft). The island consists of two roughly round portions centered on volcanic mountains and connected by a short isthmus named after the small town of Taravao, which is situated there \cite{tahitiworldatlas}.

\begin{figure}[!h]
  \centering
  \includegraphics[height=6cm]{figures/TahitiMooreaMap.png}
  \caption{Tahiti Moorea Map}
  \label{tmmap}
\end{figure}

\subsubsection{Geology}

Sub-subsection goes here, sample text.

Sample text.

\subsubsection{Climate}

Sub-subsection goes here, sample text.

Sample text.

\section{Shape Development of Tahiti-Nui}

Another Sample Section.

\section{Conclusion}

Conclusion goes here.

%% Switch back default alignment of IEEE Template section title
\patchcmd{\section}{\raggedright}{\centering}{}{}
\patchcmd{\section}{\normalsize}{\large\bfseries}{}{}


\ifCLASSOPTIONcaptionsoff
  \newpage
\fi

%% Comment below line out if you want references in a new page.
%\newpage

%% Citation files store in ref.bib
\bibliographystyle{IEEEtran}
% argument is your BibTeX string definitions and bibliography database(s)
\bibliography{IEEEabrv,ref}


\end{document}


